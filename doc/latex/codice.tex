% !TEX encoding = UTF-8
% !TEX program = pdflatex
% !TEX root = relazione.tex
% !TeX spellcheck = it_IT

\section{Note sul codice}\label{sec:codice}
\subsection{Utilizzo}\label{subsec:utilizzo}
La cartella contenente il codice è strutturata in modo tenere separati i sorgenti, i binari, il codice generato,
i file Header, scripts e altri file utili all'esecuzione.
Per questo motivo il codice viene fornito con un \textsf{Makefile} che include al suo interno i comandi
necessari alla compilazione e all'esecuzione dei test.
Per un corretto utilizzo sono necessari alcuni passi da eseguirsi sul terminale di un computer con sistema
operativo basato su Linux.
%
Il codice viene fornito sottoforma di archivio, quindi per prima cosa è necessario estrarre
i file contenuti all'interno:
\label{lst:make-passo-1}
\begin{lstlisting}[style=BashStyle]
tar -zxvf codice.tar.gz
\end{lstlisting}
per poi spostarsi all'interno della cartella appena creata:
\label{lst:make-passo-2}
\begin{lstlisting}[style=BashStyle]
cd codice/
\end{lstlisting}
%
Per compilare i sorgenti è sufficiente lanciare il comando \textsf{Make}:
\label{lst:make-passo-3}
\begin{lstlisting}[style=BashStyle]
make
\end{lstlisting}
%
Se la compilazione termina con successo, all'interno della
cartella \textsf{bin} (per ulteriori dettagli si rimanda all'apposita sotto-sezione) troveremo
l'eseguibile principale.
Tuttavia si rende necessario un passo intermedio: bisogna infatti generare le istanze su cui si andrà
poi a eseguire il binario.
Per far ciò è sufficiente digitare
\label{lst:make-passo-4}
\begin{lstlisting}[style=BashStyle]
make gen-instances
\end{lstlisting}
e, una volta completato il processo, avremmo l'insieme delle istanze, generate secondo il metodo
descritto nella sezione~\ref{sec:generazione-istanze}.

Infine, si può scegliere se eseguire gli algoritmi su tutto l'insieme di istanze:
\label{lst:make-passo-5}
\begin{lstlisting}[style=BashStyle]
make run
\end{lstlisting}
oppure se eseguire un test su una sola istanza a scelta, come ad esempio:
\label{lst:make-passo-6}
\begin{lstlisting}[style=BashStyle]
make test ARGS=instances/dcc1911_n15.tsp
\end{lstlisting}
%%%%
%%%%
\subsection{Struttura della cartella}\label{subsec:struttura}
Come già accennato, la disposizione dei file all'interno della cartella è ben strutturata e definita,
ma può risultare dispersiva a un estraneo.
Per questo motivo, viene qui fatta una panoramica dalla disposizione delle cartelle e dei file al suo interno.
Come si può vedere dalla figura~\ref{fig:struttura-codice}, all'interno della cartella pricipale troviamo:
\begin{itemize}
	\item \textsf{bin}: contiene il file binario generato dalla compilazione;
	\item \textsf{include}: all'interno sono presenti tutti i file header;
	\item \textsf{instances}: una volta generate, le istanze si troveranno qui dentro;
	\item \textsf{scripts}: si trova lo script in python che si occupa della generazione
	delle istanze;
	\item \textsf{src}: i sorgenti per gli algoritmi risolutivi delle due parti;
	\item \textsf{vlsi-dataset}: contiente i file delle istanze originali, da cui poi verranno generate tutte le altre.
\end{itemize}
%
I sorgenti sono molteplici e la loro descrizione è presente all'interno nelle prime righe;
Tuttavia, le parti fondamentali degli algoritmi richiesti sono contenute solamente in alcuni, che sono:
\begin{itemize}
	\item \textsf{CPLEXSover}: contiene la risoluzione del problema posto utilizzando le API di CPLEX (risoluzione parte 1);
	\item \textsf{GASolver} e \textsf{GAPopulation}: qui si trova la risoluzione mediante l'agoritmo genetico (parte 2);
	\item \textsf{Main}: principalmente si occupa di iterare gli algoritmi su tutte le istanze,
	ma può essere visto come esempio su come tilizzare appropriatamente le classi e i metodi per la
	risoluzione di un problema (ad esempio nella funzione \textsf{single\_test});
	\item \textsf{results.csv}: contiene i risultati delle esecuzioni sulle istanze (presente solo dopo l'esecuzione).
\end{itemize}
%
\label{fig:struttura-codice}
\begin{forest}
    for tree={font=\sffamily, grow'=0,
    folder indent=.9em, folder icons,
    edge=densely dotted}
    [main folder
      [include
          [cpxmacro.h, is file]
		  [CPLEXSolver.h, is file]
		  [GAIndividual.h, is file]
		  [GAPopulation.h, is file]
		  [GASolver.h, is file]
		  [TSPProblem.h, is file]
		  [TSPSolution.h, is file]]
      [scripts
          [gen\_instances.py, is file]]
	  [src
           [CPLEXSolver.cpp, is file]
 		  [GAIndividual.cpp, is file]
 		  [GAPopulation.cpp, is file]
 		  [GASolver.cpp, is file]
		  [main.cpp, is file]
 		  [TSPProblem.cpp, is file]
 		  [TSPSolution.cpp, is file]]
	 [vlsi-dataset
		 [bcl380.tsp, is file]
		 [dcc1911.tsp, is file]
		 [djb2036.tsp, is file]
		 [pbd984.tsp, is file]
		 [pds2566.tsp, is file]
		 [xql662.tsp, is file]]
      [Makefile, is file]
	  [results.csv, is file]
    ]
 \end{forest}
 %%%%
 %%%%
 \subsection{CPLEX}\label{subsec:cplex}
Utilizzando il Makefile non è necessario conoscere i nomi dei file o i loro parametri ed è sufficiente
utilizzare i comandi descritti in precedenza (sotto-capitolo~\ref{subsec:utilizzo}) per compilare
ed eseguire il codice.
Tuttavia potrebbe rendersi necessaria una modifica al Makefile nel caso in cui il percorso di installazione
di CPLEX sia diverso da quello indicato.
\footnote{Il Makefile presente nel codice conseganto al docente è già stato adattato per funzionare nei computer
del laboratorio. La modifica è necessaria nel caso si voglia compilare su una macchina diversa.}
In questo caso è necessario aprire il file \textsf{Makefile} con un editor di testo (nano, emacs, gedit, sublime)
e modificare il percorso a CPLEX, indicato nelle voci \textsf{CPX\_INCDIR} e \textsf{CPX\_LIBDIR}.
\label{lst:modifica-makefile}
\begin{lstlisting}[style=BashStyle]
CPX_INCDIR := <PERCORSO_CPLEX>/include/
CPX_LIBDIR := <PERCORSO_CPLEX>/lib/x86-64_linux/static_pic
\end{lstlisting}
