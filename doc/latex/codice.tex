% !TEX encoding = UTF-8
% !TEX program = pdflatex
% !TEX root = relazione.tex
% !TeX spellcheck = it_IT

\section{Note sul codice}

\subsection{Utilizzo}
La cartella contenente il codice è strutturata in modo tenere separati i sorgenti, i binari, il codice generato,
i file Header, scripts e altri file utili all'esecuzione.
Per questo motivo il codice viene fornito con un \textsf{Makefile} che include al suo interno i comandi
necessari alla compilazione e all'esecuzione dei test.
Per un corretto utilizzo sono necessari alcuni passi da eseguirsi sul terminale di un computer con sistema
operativo basato su Linux.
%
\paragraph{Inizio} Il codice viene fornito sottoforma di archivio, quindi per prima cosa è necessario estrarre
i file contenuti all'interno:
\label{code:make-passo-1}
\begin{lstlisting}[language=bash]
tar -zxvf codice.tar.gz
\end{lstlisting}
per poi spostarsi all'interno della cartella appena creata:
\label{code:make-passo-2}
\begin{lstlisting}[language=bash]
cd codice/
\end{lstlisting}
