% !TEX encoding = UTF-8
% !TEX program = pdflatex
% !TEX root = relazione.tex
% !TeX spellcheck = it_IT

\section{Note per l'utilizzo del codice}\label{sec:codice}
\subsection{Compilazione ed esecuzione}\label{subsec:utilizzo}
La cartella contenente il codice è strutturata in modo tenere separati le varie tipologie di file utili all'esecuzione.
Per questo motivo il codice viene fornito con un \textsf{Makefile} che include al suo interno i comandi
necessari alla compilazione e all'esecuzione dei test.
Per un corretto utilizzo sono necessari alcuni passi da eseguirsi sul terminale di un computer con sistema
operativo basato su Linux\footnote{I comandi descritti, così come i processi di compilazione ed esecuzione,
sono stati provati e verificati sui computer del laboratorio LabTA.}.

Il progetto viene fornito sottoforma di archivio, quindi per prima cosa è necessario estrarre i file contenuti all'interno:
\label{lst:make-passo-1}
\begin{lstlisting}[style=BashStyle]
tar -zxvf progetto-memoc-romanelli.tar.gz
\end{lstlisting}
per poi spostarsi all'interno della cartella contenente il codice:
\label{lst:make-passo-2}
\begin{lstlisting}[style=BashStyle]
cd progetto-memoc-romanelli/codice/
\end{lstlisting}
%
Per compilare i sorgenti è sufficiente lanciare il comando \textsf{Make}:
\label{lst:make-passo-3}
\begin{lstlisting}[style=BashStyle]
make
\end{lstlisting}
%
Se la compilazione termina con successo, all'interno della cartella \textsf{bin}
(per ulteriori dettagli si rimanda alla sezione~\ref{subsec:struttura}) si troverà l'eseguibile principale \textsf{main}.

Per eseguire il programma principale bisogna passare come parametro
il percorso dell'istanza di prova desiderata, che si trova nella cartella \textsf{samples};
ad esempio, se si vuole utilizzare l'istanza \textsf{dcc1911\_n025.tsp} il comando da utilizzare è:
\label{lst:make-passo-4}
\begin{lstlisting}[style=BashStyle]
bin/main "samples/dcc1911_n025.tsp"
\end{lstlisting}
%
\subsubsection{Creazione delle istanze}\label{subsubsec:creazione-istanze-note}
Se invece si volesse eseguire il programma su tutto l'insieme di istanze, generate
come scritto in sezione~\ref{sec:generazione-istanze}, per riproporre le prove fatte, è necessario utilizzare altri comandi.
Per prima cosa vanno generate le istanze:
\label{lst:make-passo-5}
\begin{lstlisting}[style=BashStyle]
make gen-instances
\end{lstlisting}
A questo punto è possibile eseguire gli algoritmi sull'insieme di istanze utilizzando il risolutore CPLEX:
\label{lst:make-passo-6}
\begin{lstlisting}[style=BashStyle]
make run-cplex
\end{lstlisting}
oppure il risolutore che utilizza l'Algoritmo Genetico:
\label{lst:make-passo-7}
\begin{lstlisting}[style=BashStyle]
make run-ga
\end{lstlisting}
%%%%
\subsection{Struttura della cartella}\label{subsec:struttura}
Come già accennato, la disposizione dei file all'interno della cartella è ben strutturata e definita.
Per questo motivo, viene proposta una veloce panoramica dalla disposizione delle cartelle e dei file al suo interno:
\begin{itemize}
	\item \textsf{bin}: contiene il file binario generato dalla compilazione;
	\item \textsf{include}: all'interno sono presenti tutti i file header;
	\item \textsf{instances}: una volta generate, le istanze si troveranno qui dentro;
	\item \textsf{samples}: contiene alcune istanze di prova;
	\item \textsf{scripts}: si trova lo script in python che si occupa della generazione
	delle istanze (per i test completi);
	\item \textsf{src}: i sorgenti per gli algoritmi risolutivi delle due parti;
	\item \textsf{vlsi-dataset}: contiente i file delle istanze originali, da cui poi verranno generate tutte le altre.
\end{itemize}
%
I sorgenti sono molteplici e una breve descrizione è presente all'interno del file nelle prime righe;
Tuttavia, le parti fondamentali degli algoritmi richiesti sono contenute solamente in alcuni, che sono:
\begin{itemize}
	\item \textsf{CPLEXSover}: contiene la risoluzione del problema posto utilizzando le API di CPLEX (risoluzione parte 1);
	\item \textsf{GASolver} e \textsf{GAPopulation}: qui si trova la risoluzione mediante l'agoritmo genetico (parte 2);
	\item \textsf{Main}: principalmente si occupa di iterare gli algoritmi su tutte le istanze,
	ma può essere visto come esempio su come utilizzare propriatamente le classi e i metodi per la
	risoluzione di un problema (ad esempio nella funzione \textsf{single\_test});
\end{itemize}
%
\subsection{CPLEX}\label{subsec:cplex}
Utilizzando il Makefile non è necessario conoscere i nomi dei file o i loro parametri ed è sufficiente
utilizzare i comandi descritti in precedenza (sotto-capitolo~\ref{subsec:utilizzo}) per compilare
ed eseguire il codice.
Tuttavia potrebbe rendersi necessaria una modifica al Makefile nel caso in cui il percorso di installazione
di CPLEX sia diverso da quello indicato\footnote{Il Makefile presente nel codice conseganto al docente è già stato adattato per funzionare nei computer
del laboratorio. La modifica è necessaria nel caso si voglia compilare su una macchina diversa.}.
In questo caso è necessario aprire il file \textsf{Makefile} con un editor di testo (nano, emacs, gedit, sublime)
e modificare il percorso a CPLEX, indicato nelle voci \textsf{CPX\_INCDIR} e \textsf{CPX\_LIBDIR}.
\label{lst:modifica-makefile}
\begin{lstlisting}[style=BashStyle]
CPX_INCDIR := <PERCORSO_CPLEX>/include/
CPX_LIBDIR := <PERCORSO_CPLEX>/lib/x86-64_linux/static_pic
\end{lstlisting}
