% !TEX encoding = UTF-8
% !TEX program = pdflatex
% !TEX root = relazione.tex
% !TeX spellcheck = it_IT

\section{Generazione istanze}\label{sec:generazione-istanze}
È ovvio come per prima cosa sia necessario creare delle istanze per il problema che sia andrà ad analizzare.
Un'istanza di un problema di foratura (come descritto nella prima parte della consegna) può essere rappresentata
da un grafo dove i nodi sono i fori e gli archi con relativo peso sono gli spostamenti e i costi che la trivella
deve effettuare per muoversi.
Rappresentando questo tramite una matrice dei costi, il problema ora è come generare tali nodi e distanze;
i fori nei pannelli perforati, infatti, non sono distribuiti casualmente sullo spazio ma seguono delle
logiche intrinseche del problema che si fa a risolvere.

Per cercare, quindi, di riprodurre almeno in parte questa distribuzione si è utilizzato una collezione di istanze
per TSP basate su un data-set di VLSI, fornite da Andre Roh dell'Istituto di Ricerca per la Matematica Discreta
dell'Università di Bonn, consultabili gratuitamente sul sito web ufficiale
\footnote{\url{http://www.math.uwaterloo.ca/tsp/vlsi/index.html}}.
Un esempio di tale istanze si può vedere in figura~\ref{fig:istanza-tsp-XQF131}.
%
\begin{figure}[!h]
\begin{center}
	\includegraphics[scale=.3]{XQF131}
{\scriptsize \caption{Esempio di problema TSP con 131 città.}
\label{fig:istanza-tsp-XQF131}}
\end{center}
\end{figure}
%
\paragraph{Generazione istanze pseudo-casuali}
Queste istanze hanno però un numero di nodi molto elevato rispetto alle necessità
e alle risorse di questa analisi.
Inoltre, data la natura probabilistica dell'algoritmo previsto per la soluzione della parte 2 della consegna,
è necessario eseguire le soluzioni proposte su molte istanze diverse.

Per questo motivo è stato implementato un generatore di istanze pseudo-casuali che, a partire dalle istanze TSP
descritte in precedenza, generi delle sotto-istanze di dimensioni minori.
Dati i punti dell'istanza originale, il generatore prima estrare casualmente un sotto-insieme di questi
di dimensione variabile per poi calcolare le distanze (futuri costi) sui nuovi punti.

Partendo da un'istanza originale, il generatore crea delle nuove istanze con dimensione a partire da 5 fino ad arrivare
a 40 con un passo di 5.
Successivamente ne vengono create altre a partire da 50 con un passo di 10 fino a 80.
Infine le ultime due vengono generate con dimensione 100 e 120 (da 100 a 120 con passo 20).
Il processo viene ripetuto per quattro istanze originali, generando quindi 57 nuove istanze pseudo-casuali.
Il numero del passo va via via aumentando per permettere di eseguire i test in tempi più ragionevoli
(basti pensare che per ogni istanza pseudo-casuale l'algoritmo CPLEX  ha tempo limite pari a 5 minuti,
arrivando quindi a, potenzialmente, quasi 5 ore d'esecuzione per la prima consegna).

Per l'implementazione è stato creato appositamente uno script scritto nel linguaggio Python.
%%%
