% !TEX encoding = UTF-8
% !TEX program = pdflatex
% !TEX root = relazione.tex
% !TeX spellcheck = it_IT

\section{Algoritmo Genetico}\label{sec:ga}
Come da consegna, la seconda parte prevedeva l'implementazione di un algortimo ad-hoc, scegliendo fra le varie
metaeuristica presenti nelle dispense del corso;
fra queste, si è scelta la classe dei metodi basati su popolazione e, in particolare, gli Algoritmi Genetici,
molto diffusi grazie alla loro adattabilità e semplicità implementativa.
%%%%
%%%%
\subsection{Operatori genetici}\label{subsec:operatori-genetici}
Trattandosi di una metaeuristica, lo schema principale è generale e lascia molto spazio di manovra.
Si inzia con la \textit{codifica delle soluzioni}, per poi definire i diversi \textit{operatori genetici}:
\begin{itemize}
	\item la generazione di un insieme di soluzioni (popolazione iniziale);
	\item funzione che valuta la fitness di una soluzione;
	\item operatori di ricombinazione;
	\item operatori di passaggio generazionale.
\end{itemize}
Per la descrizione della metaeuristica si rimanda alle dispsense, mentre qui si evenzieranno
solo le scelte progettuali attuate.
%%%%
%%%
\subsubsection{Codifica degli individui}\label{subsubsec:codifica-individui}
Un individuo della popolazione è una seguenza di $N+1$ geni, dove $N$ è la dimensione del problema e
dove ciascun gene in posizione $i$ indica un nodo (o foro per la trivella) da visitare in posizione $i$
nel ciclo hamiltoniano (il percorso che dovrà fare la trivella).
La sequenza ha dimensione maggiore di $1$ rispetto alla dimensione del problema per codificare direttamente
il vincolo che la trivella ritorni al punto di partenza; l'ultimo gene, infatti, avrà sempre valore uguale
al primo, cioè $0$.
%%%
\subsubsection{Popolazione iniziale}\label{subsubsec:popolazione-inziale}
La popolazione iniziale viene creata generando delle soluzioni (individui) in modo pseudo-greedy, ovvero
incrementalmente a partire dal nodo inziale, scegliendo il successivo casualmente fra quelli ancora da visitare,
ma dando più probabilità a quelli con costo minore.
Questo per permettere sia una diversificazione degli individui (parte probabilistica) mantentendo tuttavia
una convergenza abbastanza veloce (scelte pesate).

La dimensione è specificata tramite un parametro ed è direttamente proporzionale alla dimensione del problema;
è ragionevole pensare, infatti, che istanze maggiori abbiano un spazio delle soluzioni maggiore.
%%%%
\subsubsection{Funzione di fitness}\label{subsubsec:funzione-fitness}
La funzione di fitness serve a dare una misura quantitiva della bontà di un individuo e guida
molti dei processi evolutivi dell'algoritmo.
Per questo motivo, è stata scelta la funzione obbiettivo del modello, ovvero il costo del ciclo della soluzione.
%%%%
\subsubsection{Operatori di selezione}\label{subsubsec:operatori-selezione}
Gli operatori di selezione scelgono tra la popolazione corrente gli invidui che partecipano ai processi riproduttivi.
La scelta dev'essere in parte pesata sugli invidui migliori, in modo da trasmettere le migliori caratteristiche
alla progenie, ma anche una componente casuale che permetta di convergere più lentamente e introdurre caratteristiche
diverse che potrebbero rivelarsi anch'esse migliori.

Sono stati proposti diversi metodi per combinare questi due aspetti, ma la scelta è ricaduta sul metodo
chiamato \textit{Torneo-n}.
Vengono quindi prima scelti $n$ individui e poi fra questi estratto il miglior candidato; il processo viene ripetuto
una seconda volta in modo da ottenere due genitori.

Il valore degli $n$ individui è impostato sul $20\%$ della popolazione.
%%%
\subsubsection{Operatori di ricombinazione}\label{subsubsec:operatori-ricombinazione}
Gli operatori di ricombinazione, chimati in inglese di \textit{crossover},generano uno o più figli a partire dai genitori.
Come già detto, i genitori in questo caso sono due (come avviene nella maggioranza dei casi) e viene generato
un solo individuo figlio utilizzando il metodo noto come \textit{cut-cut crossover}.
Il metodo nasce dall'ipotesi che geni vicini fra loro controllino caratteristiche tra loro correlate e, quindi,
affinché i figli preservino tali caratteristiche sia necessario trasmettere blocchi di geni contigui.
In dettaglio, vengono definiti casualmente 2 punti di taglio (e quindi sarà un 2-cut crossover) e il figlio si ottiene
copiano i blocchi dei genitori alternativamente.
%%
\begin{figure}[!h]
\begin{center}
	\includegraphics[scale=.6]{2-cut-point-crossover}
{\scriptsize \caption{Esempio di 2-cut-point crossover con un figlio.}
\label{fig:2-cut-point-crossover}}
\end{center}
\end{figure}
%%
Per preservare l'ammissibilità dei figli (e generare quindi soluzioni ammissibili) è stato implementata
una variante del metodo appena descritto, ovvero l'\textit{order crossover}.
In questa tecnica, il figlio riporta le parti esterne dal primo genitore mentre i restanti geni (blocco interno del figlio)
si ottengono copiando i geni mancanti nell'ordine in cui appiaiono nel secondo genitore.
Si noti che questo metodo preserva il vincolo che il primo e l'ultimo gene siano entrambi pari a 0.
%%%
\subsubsection{Mutazione}\label{subsubsec:mutazione}
Per evitare il fenomeno chiamato assorbimento genetico (convergenza casuale di uno o più geni verso lo stesso valore),
l'operatore di mutazinone modifica casualmente il valore di alcuni geni, scelti casualmente anch'essi.

Anche in questo, caso è stato implementato un metodo che preserva l'ammissibilità, chiamato \textit{mutazione per inversione},
dove sono generati casualmente due punti della sequenza e si inverte la sottosequenza tra questi.

La probabilità di mutazione è modificabile tramite un apposito parametro ed è solitamente molto bassa.
%%%
\subsubsection{Sostituzione della popolazione}\label{subsubsec:sostituzione-popolazione}
A questo punto si ha un nuovo insieme di individui che va sostituito alla generazione corrente.
Diverse tecniche sono state proposte e si è scelto di implementare la politica chiamata \textit{selezione dei migliori}.
Si mantengono nella popolazione corrente gli N individui migliori tra gli $N + R$ (con $N$ dimensione della popolazione corrente
e $R$ i nuovi individui generati), effettuando la selezione utilizzando il metodo Montecarlo con probabilità
proporzionale al valore di fitness.
%%%
\subsubsection{Criterio di arresto}\label{subsubsec:criterio-arresto}
% TODO: se si cambia il codice aggiornare qua
I possibili criteri per l'arresto sono diversi e sono possibili anche combinazioni fra questi.
In questo caso si è scelto un metodo ibrido che prevede un numero massimo di iterazioni (evoluzioni
della popolazione) e tempo massimo d'esecuzione, entrambi specificabili come parametri all'avvio dell'algoritmo.

Il tempo limite permette un confronto più semplice con il metodo CPLEX (i tempi limite sono infatti uguali)
ed evita che l'algoritmo richieda troppo tempo quando la dimensione del problema è molto grande;
il numero massimo di iterazioni permette invece di limitare il tempo d'esecuzione
quando le istanze hanno una bassa dimensione.
%%%%
%%%%
\subsection{Taratura dei parametri}\label{subsec:parametri}
Come detto in precedenza, gli algoritmi genetici dipendono molto dalla loro configurazione, ossia
dai valori che si sceglie di dare ai vari parametri.
Nel caso in esame, quattro sono i parametri su cui si è andato ad agire:
\begin{itemize}
	\item \textsf{time\_limit}: tempo limite per l'evoluzione della popolazione;
	\item \textsf{ga\_iteration\_limit}: numero massimo di iterazioni o, in altre parole, numero massimo di evoluzioni
	della popolazione;
	\item \textsf{ga\_population\_size\_factor}: la popolazione iniziale avrà dimensione pari a questo valore
	moltiplicato per la dimensione del problema;
	\item \textsf{ga\_mutation\_probability}: probabilità che un individuo subisca una mutazione.
\end{itemize}
%
\paragraph{Limite di tempo}
Dovendo iterare l'esecuzione dell'algoritmo su diverse istanze e dovendone confrontare i risultati
con l'implementazione con CPLEX, è stato necessario introdurre un limite massimo all'esecuzione.
Il valore è impostabile secondo le preferenze del momento e al problema che si sta considerando,
a seconda del fatto che si voglia una soluzione migliore oppure maggior rapidità d'esecuzione.
Per motivi logistici e di tempistiche, il valore scelto è pari a 5 minuti.
%
\paragraph{Numero massimo di iterazioni}
Come per il limite di tempo, questo valore è stato deciso sulla base di diverse prove effettuate
e cerca di bilanciare una buona evoluzione della popolazione inziale e ovvi costi computazionali.
Il valore scelto di 1000 iterazioni sembra essere un buon compromesso.
%
\paragraph{Dimensione della popolazione iniziale}
In questo caso non è possibile specificare direttamente la dimensione iniziale, bensì si imposta
un fattore di moltiplicazione che verrà combinato con la dimensione del problema.
Visto che, nelle istanze su cui si andrà poi a eseguire l'algoritmo, questo valore
potrà assumere una gamma di valori abbastanza ampia (da 5 a 100), impostare un numero fisso
per la dimensione iniziale sarebbe risultato troppo abbondante per le piccole istanze
e troppo riduttivo per quelle grandi.

Per determinare il valore del fattore di moltiplicazione si sono fatte diverse prove e si è visto
che, naturalmente, all'aumentare di tale valore l'algoritmo terminava con una soluzione migliore
(un numero maggiore di invidui implica una maggior probabilità che la soluzione ottima sia fra questi
o che le `caratteristiche genetiche' di queste siano presenti) ma ovviamente aumentava di molto tempo d'esecuzione.
Considerato che la fase di creazione della popolazione e successivamente la sua sostituzione sono le fasi
computazionalmente più onerose, si è preferito tenere basso il fattore di moltiplicazione, preferendo
un'esecuzione più snella a fronte di una soluzione leggermente meno ottimale.

Il valore finale scelto è pari a 3, producendo quindi una popolazione iniziale di 3 volte superiore
alla dimensione del problema.
%
\paragraph{Probabilità della mutazione}
Una probabilità di mutazione alta aumenta le probabilità di esplorare più aree dello spazio di ricerca
ma rallenta la convergenza all'ottimo globale; d'altre parte, un valore basso potrebbe portare a una prematura convergenza (ottimo locale).
Seguendo anche le indicazione fornite sulle dispense, il valore finale scelto è $0,05$.
%%%%
%%%%
\subsection{Risultati}\label{subsec:risultati}
% TODO -
%%%%
