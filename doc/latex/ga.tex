% !TEX encoding = UTF-8
% !TEX program = pdflatex
% !TEX root = relazione.tex
% !TeX spellcheck = it_IT

\section{Algoritmo Genetico}\label{sec:GA}
Come da consegna, la seconda parte prevedeva l'implementazione di un algortimo ad-hoc, scegliendo fra le varie
metaeuristica presenti nelle dispense del corso;
fra queste, si è scelta la classe dei metodi basati su popolazione e, in particolare, gli Algoritmi Genetici,
molto diffusi grazie alla loro adattabilità e semplicità implementativa.
%%%%
%%%%
\subsection{Operatori genetici}\label{subsec:operatori-genetici}
Trattandosi di una metaeuristica, lo schema principale è generale e lascia molto spazio di manovra.
Si inzia con la \textit{codifica delle soluzioni}, per poi definire i diversi \textit{operatori genetici}:
\begin{itemize}
	\item la generazione di un insieme di soluzioni (popolazione iniziale);
	\item funzione che valuta la fitness di una soluzione;
	\item operatori di ricombinazione;
	\item operatori di passaggio generazionale.
\end{itemize}
Per la descrizione della metaeuristica si rimanda alle dispsense, mentre qui si evenzieranno
solo le scelte progettuali attuate.
%%%%
%%%
\subsubsection{Codifica degli individui}\label{subsubsec:codifica-individui}
Un individuo della popolazione è una seguenza di $N+1$ geni, dove $N$ è la dimensione del problema e
dove ciascun gene in posizione $i$ indica un nodo (o foro per la trivella) da visitare in posizione $i$
nel ciclo hamiltoniano (il percorso che dovrà fare la trivella).
La sequenza ha dimensione maggiore di $1$ rispetto alla dimensione del problema per codificare direttamente
il vincolo che la trivella ritorni al punto di partenza; l'ultimo gene, infatti, avrà sempre valore uguale
al primo, cioè $0$.
%%%
\subsubsection{Popolazione iniziale}\label{subsubsec:popolazione-inziale}
La popolazione iniziale viene creata generando delle soluzioni (individui) in modo pseudo-greedy, ovvero
incrementalmente a partire dal nodo inziale, scegliendo il successivo casualmente fra quelli ancora da visitare,
ma dando più probabilità a quelli con costo minore.
Questo per permettere sia una diversificazione degli individui (parte probabilistica) mantentendo tuttavia
una convergenza abbastanza veloce (scelte pesate).

La dimensione è specificata tramite un parametro ed è direttamente proporzionale alla dimensione del problema;
è ragionevole pensare, infatti, che istanze maggiori abbiano un spazio delle soluzioni maggiore.
%%%%
\subsubsection{Funzione di fitness}\label{subsubsec:funzione-fitness}
La funzione di fitness serve a dare una misura quantitiva della bontà di un individuo e guida
molti dei processi evolutivi dell'algoritmo.
Per questo motivo, è stata scelta la funzione obbiettivo del modello, ovvero il costo del ciclo della soluzione.
%%%%
\subsubsection{Operatori di selezione}\label{subsubsec:operatori-selezione}


%%%
