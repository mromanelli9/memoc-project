% !TEX encoding = UTF-8
% !TEX program = pdflatex
% !TEX root = relazione.tex
% !TeX spellcheck = it_IT

\section{Introduzione}

L'obbiettivo del progetto è quello di implementare il noto problema combinatorio del commesso viaggiatore (Traveler Salesman Problem), d'ora in avanti TSP, in due modalità differenti
per poi valutarne le differenze.
In particolare, la prima parte consiste nell'implementazione del problema utilizzando le API di CPLEX e trovare la dimensione massima del problema risolvibile entro un certo intervallo
di tempo (fino a 1 secondo, fino a 10 secondi, eccetera).
La seconda parte, invece, si chiede di studiare e implementare un algoritmo di ottimizzazione ad-hoc, utilizzando una qualsiasi meta-euristica vista a lezione.
Fatto ciò si richiede di testare l'implementazione, presentarne i costi computazionali e compararne i risultati con il metodo utilizzato nella prima parte.

% TODO: mettere i riferimenti alle sezioni
La presente relazione procederà come segue: nella prossima sezione verrà presentata la generazione delle istanze esponendo le caratteristiche di queste.
La sezione 3 presenterà la prima parte del progetto, partendo dalla soluzione con CPLEX e finendo con i risultati ottenuti.
La sezione 4, invece, esporrà la seconda parte del progetto.
Nella sezione 5 verrà introdotto brevemente il metodo di utilizzo del codice, come ad esempio compilazione ed esecuzione.
Per concludere, nella sezione 6 verranno presentate le conclusione finali.