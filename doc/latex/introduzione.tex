% !TEX encoding = UTF-8
% !TEX program = pdflatex
% !TEX root = relazione.tex
% !TeX spellcheck = it_IT

\section{Introduzione}\label{sec:introduzione}

L'obbiettivo del progetto è quello di implementare il noto problema combinatorio del commesso viaggiatore (Traveler Salesman Problem), d'ora in avanti TSP, in due modalità differenti
per poi valutarne le differenze.
In particolare, la prima parte consiste nell'implementazione del problema utilizzando le API di CPLEX e trovare la dimensione massima del problema risolvibile entro un certo intervallo
di tempo (fino a 1 secondo, fino a 10 secondi, eccetera).
La seconda parte, invece, si chiede di studiare e implementare un algoritmo di ottimizzazione ad-hoc, utilizzando una qualsiasi meta-euristica vista a lezione.
Fatto ciò si richiede di testare l'implementazione, presentarne i costi computazionali e compararne i risultati con il metodo utilizzato nella prima parte.

La presente relazione procederà come segue.
Per prima cosa nella sezione~\ref{sec:generazione-istanze} verrà presentato il problema della generazione delle istanze.
La sezione~\ref{sec:cplex} illustrerà la prima parte del progetto, partendo dalla soluzione con CPLEX e finendo con i risultati ottenuti.
La sezione 4~\ref{sec:ga}, invece, esporrà la seconda parte del progetto.
Per concludere, nella sezione~\ref{sec:conclusioni} verranno presentate le conclusione finali.
In appendice, alla sezione~\ref{sec:codice} si può trovare una panoramica del codice e gli elementi base per il suo utilizzo.
