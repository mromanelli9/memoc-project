% !TEX encoding = UTF-8
% !TEX program = pdflatex
% !TEX root = relazione.tex
% !TeX spellcheck = it_IT

\section{Conclusioni}\label{sec:conclusioni}
Come si è visto, in generale l'algoritmo GA ha trovato soluzioni in un tempo inferiore rispetto al
risolutore CPLEX che, tuttavia, ha trovato spesso soluzioni migliori;
inoltre le soluzioni trovate (sotto certe condizioni) con CPLEX sono garantite ottime
mentre non è così con l'algoritmo genetico.
Si tratta quindi di decidere, in base al problma che si vuole risolvere, se si è disposti
ad avere soluzioni buone ma non garantite ottime con un costo computazionale inferiore,
oppure la sicurezza di soluzioni ottime ma con un tempo d'esecuzione nettamente superiore.

Un appunto va anche fatto alle istanze su cui si sono effettuate le prove.
Seppur generate da istanze TSP reali e con particolari accorgimenti per renderle
più veritiere, le istanze pseudo-casuali sono tuttavia frutto di una generazione
casuale e pertanto potrebbero non rappresentare a pieno le caratteristiche di un problema
TSP (o di perforazione) reale.
In aggiunta a ciò, la dimensione dei problemi rappresentate da queste, ossia il numero di fori da perforare,
è molto limitata considerato che i problemi reali sono composti da migliaia di nodi.
